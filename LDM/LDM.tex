% Options for packages loaded elsewhere
% Options for packages loaded elsewhere
\PassOptionsToPackage{unicode}{hyperref}
\PassOptionsToPackage{hyphens}{url}
\PassOptionsToPackage{dvipsnames,svgnames,x11names}{xcolor}
%
\documentclass[
  letterpaper,
  DIV=11,
  numbers=noendperiod]{scrartcl}
\usepackage{xcolor}
\usepackage{amsmath,amssymb}
\setcounter{secnumdepth}{-\maxdimen} % remove section numbering
\usepackage{iftex}
\ifPDFTeX
  \usepackage[T1]{fontenc}
  \usepackage[utf8]{inputenc}
  \usepackage{textcomp} % provide euro and other symbols
\else % if luatex or xetex
  \usepackage{unicode-math} % this also loads fontspec
  \defaultfontfeatures{Scale=MatchLowercase}
  \defaultfontfeatures[\rmfamily]{Ligatures=TeX,Scale=1}
\fi
\usepackage{lmodern}
\ifPDFTeX\else
  % xetex/luatex font selection
\fi
% Use upquote if available, for straight quotes in verbatim environments
\IfFileExists{upquote.sty}{\usepackage{upquote}}{}
\IfFileExists{microtype.sty}{% use microtype if available
  \usepackage[]{microtype}
  \UseMicrotypeSet[protrusion]{basicmath} % disable protrusion for tt fonts
}{}
\makeatletter
\@ifundefined{KOMAClassName}{% if non-KOMA class
  \IfFileExists{parskip.sty}{%
    \usepackage{parskip}
  }{% else
    \setlength{\parindent}{0pt}
    \setlength{\parskip}{6pt plus 2pt minus 1pt}}
}{% if KOMA class
  \KOMAoptions{parskip=half}}
\makeatother
% Make \paragraph and \subparagraph free-standing
\makeatletter
\ifx\paragraph\undefined\else
  \let\oldparagraph\paragraph
  \renewcommand{\paragraph}{
    \@ifstar
      \xxxParagraphStar
      \xxxParagraphNoStar
  }
  \newcommand{\xxxParagraphStar}[1]{\oldparagraph*{#1}\mbox{}}
  \newcommand{\xxxParagraphNoStar}[1]{\oldparagraph{#1}\mbox{}}
\fi
\ifx\subparagraph\undefined\else
  \let\oldsubparagraph\subparagraph
  \renewcommand{\subparagraph}{
    \@ifstar
      \xxxSubParagraphStar
      \xxxSubParagraphNoStar
  }
  \newcommand{\xxxSubParagraphStar}[1]{\oldsubparagraph*{#1}\mbox{}}
  \newcommand{\xxxSubParagraphNoStar}[1]{\oldsubparagraph{#1}\mbox{}}
\fi
\makeatother


\usepackage{longtable,booktabs,array}
\usepackage{calc} % for calculating minipage widths
% Correct order of tables after \paragraph or \subparagraph
\usepackage{etoolbox}
\makeatletter
\patchcmd\longtable{\par}{\if@noskipsec\mbox{}\fi\par}{}{}
\makeatother
% Allow footnotes in longtable head/foot
\IfFileExists{footnotehyper.sty}{\usepackage{footnotehyper}}{\usepackage{footnote}}
\makesavenoteenv{longtable}
\usepackage{graphicx}
\makeatletter
\newsavebox\pandoc@box
\newcommand*\pandocbounded[1]{% scales image to fit in text height/width
  \sbox\pandoc@box{#1}%
  \Gscale@div\@tempa{\textheight}{\dimexpr\ht\pandoc@box+\dp\pandoc@box\relax}%
  \Gscale@div\@tempb{\linewidth}{\wd\pandoc@box}%
  \ifdim\@tempb\p@<\@tempa\p@\let\@tempa\@tempb\fi% select the smaller of both
  \ifdim\@tempa\p@<\p@\scalebox{\@tempa}{\usebox\pandoc@box}%
  \else\usebox{\pandoc@box}%
  \fi%
}
% Set default figure placement to htbp
\def\fps@figure{htbp}
\makeatother





\setlength{\emergencystretch}{3em} % prevent overfull lines

\providecommand{\tightlist}{%
  \setlength{\itemsep}{0pt}\setlength{\parskip}{0pt}}



 


\KOMAoption{captions}{tableheading}
\makeatletter
\@ifpackageloaded{caption}{}{\usepackage{caption}}
\AtBeginDocument{%
\ifdefined\contentsname
  \renewcommand*\contentsname{Table of contents}
\else
  \newcommand\contentsname{Table of contents}
\fi
\ifdefined\listfigurename
  \renewcommand*\listfigurename{List of Figures}
\else
  \newcommand\listfigurename{List of Figures}
\fi
\ifdefined\listtablename
  \renewcommand*\listtablename{List of Tables}
\else
  \newcommand\listtablename{List of Tables}
\fi
\ifdefined\figurename
  \renewcommand*\figurename{Figure}
\else
  \newcommand\figurename{Figure}
\fi
\ifdefined\tablename
  \renewcommand*\tablename{Table}
\else
  \newcommand\tablename{Table}
\fi
}
\@ifpackageloaded{float}{}{\usepackage{float}}
\floatstyle{ruled}
\@ifundefined{c@chapter}{\newfloat{codelisting}{h}{lop}}{\newfloat{codelisting}{h}{lop}[chapter]}
\floatname{codelisting}{Listing}
\newcommand*\listoflistings{\listof{codelisting}{List of Listings}}
\makeatother
\makeatletter
\makeatother
\makeatletter
\@ifpackageloaded{caption}{}{\usepackage{caption}}
\@ifpackageloaded{subcaption}{}{\usepackage{subcaption}}
\makeatother
\usepackage{bookmark}
\IfFileExists{xurl.sty}{\usepackage{xurl}}{} % add URL line breaks if available
\urlstyle{same}
\hypersetup{
  pdftitle={卒業研究1 第14回報告レポート},
  pdfauthor={後藤 健一郎},
  colorlinks=true,
  linkcolor={blue},
  filecolor={Maroon},
  citecolor={Blue},
  urlcolor={Blue},
  pdfcreator={LaTeX via pandoc}}


\title{卒業研究1 第14回報告レポート}
\author{後藤 健一郎}
\date{2026-01-12}
\begin{document}
\maketitle

\renewcommand*\contentsname{Table of contents}
{
\hypersetup{linkcolor=}
\setcounter{tocdepth}{3}
\tableofcontents
}

\subsection{要旨}\label{ux8981ux65e8}

本報告では、Latent Diffusion Model (LDM)
の再現実験を中心に、計算環境およびデータセット差が学習挙動に与える影響を調査した。LDM
は VAE
により画像を潜在空間へ圧縮し、潜在空間で拡散モデルを学習することで高解像度生成を効率化する。サンプリングには
DDIM を用い、生成時間と品質のトレードオフを検討した。CelebA-HQ と
CIFAR-10 を対象とした結果、LDM は 128px
以上の解像度で有効に機能する一方、低解像度では潜在空間の情報欠落が顕著であり画質が劣化した。今後は解像度に応じた圧縮率の最適化と、低解像度での代替手法の検討が必要である。

\subsection{1. 課題と目的}\label{ux8ab2ux984cux3068ux76eeux7684}

本課題の目的は、LDM の再現実験を通じて以下を検証することである。

\begin{itemize}
\tightlist
\item
  LDM が高解像度生成に有効であることの再確認
\item
  計算環境(Colab と Mac)やデータセット差による学習挙動の比較
\item
  DDIM を用いた効率的なサンプリングの適用可能性の検討
\end{itemize}

\subsection{2. 用いたデータ}\label{ux7528ux3044ux305fux30c7ux30fcux30bf}

\begin{itemize}
\tightlist
\item
  \textbf{CelebA-HQ}: 顔画像データセット(128px / 256px で実験)
\item
  \textbf{CIFAR-10}: 32px の低解像度画像データセット
\end{itemize}

\subsection{3. 用いた手法}\label{ux7528ux3044ux305fux624bux6cd5}

\subsubsection{3.1 Latent Diffusion Model
(LDM)}\label{latent-diffusion-model-ldm}

LDM は VAE
により画像を潜在空間に圧縮し、潜在空間上で拡散モデルを学習する。これにより、高解像度画像に対する計算量・メモリ負荷を低減しつつ生成品質を維持できる。

\begin{itemize}
\tightlist
\item
  潜在空間学習: VAE によるエンコード・デコード
\item
  拡散モデル: U-Net + Attention
\item
  ノイズスケジュール: cosine, 時間ステップ数 T=1000
\end{itemize}

潜在空間サイズの例を以下に示す。

\begin{itemize}
\tightlist
\item
  128px, 圧縮率 f=4 → 32×32 latent
\item
  256px, 圧縮率 f=8 → 32×32 latent
\end{itemize}

\subsubsection{3.2 DDIM
サンプリング}\label{ddim-ux30b5ux30f3ux30d7ux30eaux30f3ux30b0}

DDIM
は少ないステップ数でも画質を保ちやすく、決定論的サンプリングで比較実験に適するため、LDM
の生成に採用した。

\subsection{4. 実験設定}\label{ux5b9fux9a13ux8a2dux5b9a}

表1に主要な実験設定を示す。

\begin{longtable}[]{@{}lrr@{}}
\toprule\noalign{}
項目 & Colab & Mac \\
\midrule\noalign{}
\endhead
\bottomrule\noalign{}
\endlastfoot
GPU/CPU & T4 GPU & M4 chip \\
データセット & CelebA-HQ & CelebA-HQ \\
解像度 & 128px & 256px \\
圧縮率 & f=8 & f=8 \\
Latent & 16×16 & 32×32 \\
VAE & diffusers AutoencoderKL & stabilityai/sd-vae-ft-mse \\
Batch & 64 & 8 \\
学習率 & 2e-4 & 2e-4 \\
Timesteps & 1000 & 1000 \\
Schedule & cosine & cosine \\
Sampler & DDIM & DDIM \\
\end{longtable}

また、128px 画像では比較のために圧縮率 f=4 の設定も一部試行した。

\subsection{5.
実験結果と考察}\label{ux5b9fux9a13ux7d50ux679cux3068ux8003ux5bdf}

\subsubsection{5.1 CelebA-HQ}\label{celeba-hq}

\begin{itemize}
\tightlist
\item
  顔の輪郭や髪型などの大域構造は復元される傾向が確認された。
\item
  128px + f=4 では表情などの局所構造が比較的保持されやすかった。
\item
  256px + f=8 は計算負荷が高く、学習が遅い傾向にあった。
\end{itemize}

\paragraph{考察}\label{ux8003ux5bdf}

CelebA-HQ のように 128px
以上の解像度では、潜在空間が十分に確保できるため LDM
が有効に機能する。一方で計算資源がボトルネックとなり、長時間学習が必要であった(Colab
約6時間、Mac 約13時間)。

\subsubsection{5.2 CIFAR-10}\label{cifar-10}

\begin{itemize}
\tightlist
\item
  形状の識別はできるものの、\textbf{溶けたような画像}となり画質が改善されなかった。
\item
  Loss は低下するが視覚的品質に結びつかない傾向が見られた。
\end{itemize}

\paragraph{考察}\label{ux8003ux5bdf-1}

CIFAR-10 は 32px と低解像度であるため、VAE
による圧縮後の潜在空間が極端に小さくなり、情報欠落が顕著となった。例えば
f=8 では 4×4 latent
となり、潜在空間での拡散学習だけでは欠落情報を復元できず、画質低下の要因となった。低解像度では
pixel-space diffusion の方が適している可能性が高い。

\subsection{6. まとめ}\label{ux307eux3068ux3081}

\begin{itemize}
\tightlist
\item
  LDM は 128px 以上の画像で有効である。
\item
  圧縮率は潜在空間が 32×32 以上になるように選ぶことが重要である。
\item
  低解像度では潜在空間の情報欠落が支配的となるため、pixel-space
  diffusion など代替手法の検討が必要である。
\end{itemize}

\subsection{7.
やり残した課題・工夫点・苦労した点}\label{ux3084ux308aux6b8bux3057ux305fux8ab2ux984cux5de5ux592bux70b9ux82e6ux52b4ux3057ux305fux70b9}

\subsubsection{やり残した課題}\label{ux3084ux308aux6b8bux3057ux305fux8ab2ux984c}

\begin{itemize}
\tightlist
\item
  解像度や圧縮率を体系的に変化させた網羅的な比較
\item
  低解像度向けの VAE 設計や pixel-space diffusion との直接比較
\item
  学習安定化のためのハイパーパラメータ最適化
\end{itemize}

\subsubsection{工夫した点}\label{ux5de5ux592bux3057ux305fux70b9}

\begin{itemize}
\tightlist
\item
  計算環境を Colab と Mac の2種類で比較し、資源差の影響を把握した
\item
  DDIM を採用し、生成ステップ数と品質のトレードオフを検討した
\end{itemize}

\subsubsection{苦労した点}\label{ux82e6ux52b4ux3057ux305fux70b9}

\begin{itemize}
\tightlist
\item
  高解像度設定では学習時間が長く、試行回数を確保するのが難しかった
\item
  低解像度データに対して潜在空間の情報欠落が大きく、改善策の検討に時間を要した
\end{itemize}

\subsection{参考文献}\label{ux53c2ux8003ux6587ux732e}

\begin{enumerate}
\def\labelenumi{\arabic{enumi}.}
\tightlist
\item
  Ho, J., Jain, A., \& Abbeel, P. (2020). \emph{Denoising Diffusion
  Probabilistic Models}. NeurIPS.
\item
  Song, J., Meng, C., \& Ermon, S. (2020). \emph{Denoising Diffusion
  Implicit Models}. ICLR.
\item
  Rombach, R., Blattmann, A., Lorenz, D., Esser, P., \& Ommer, B.
  (2022). \emph{High-Resolution Image Synthesis with Latent Diffusion
  Models}. CVPR.
\end{enumerate}




\end{document}
